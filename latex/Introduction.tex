

\documentclass[11pt]{article}
\usepackage{amsmath}
\usepackage{epsfig}
\usepackage{graphicx}
\graphicspath{{../images/}}
\usepackage{amsfonts}
\usepackage{amssymb}
\usepackage{verbatim}
\usepackage{caption}
\usepackage{subcaption}
\usepackage[authoryear]{natbib}


%\textwidth 15cm \textheight 20cm

\addtolength{\oddsidemargin}{-.5in}
\addtolength{\evensidemargin}{-.5in}
\addtolength{\headheight}{-.5in}
\addtolength{\textheight}{1.1in}
\addtolength{\textwidth}{1.0in}


\setlength{\parskip}{.2cm}
\setlength{\parindent}{0cm}

\renewcommand{\baselinestretch}{1.2}
\renewcommand{\theequation}{\thesection.\arabic{equation}}
\renewcommand{\thefootnote}{\fnsymbol{footnote}}


\numberwithin{equation}{section}


%\theoremstyle{definition}
\newtheorem{defn}{Definition}[section]
\newtheorem{hyp}{Hyp.}
\newtheorem{assm}{Assumption}

%\theoremstyle{plain}
\newtheorem{thm}[defn]{Theorem}
\newtheorem{main}{Main Theorem}
\newtheorem{cor}[defn]{Corollary}
\newtheorem{lem}[defn]{Lemma}
\newtheorem{prop}[defn]{Proposition}
\newtheorem{conj}{Conjecture}

%\theoremstyle{remark}
\newtheorem{rem}[defn]{Remark}
\newtheorem{ex}{Example}
\newtheorem{oldthm}{ }
\newtheorem{fact}{Fact}
\newtheorem{ques}{Question}


\newcommand{\D}[1]{{\mathbb#1}}% Doubled -Blackboard bold - caps only
\newcommand{\NN}{{\D{N}}}
\newcommand{\RR}{{\D{R}}}
\newcommand{\CC}{{\D{C}}}
\newcommand{\Rn}{{\D{R}^n}}
\newcommand{\QQ}{{\D{Q}}}
\newcommand{\TT}{{\D{T}}}
\newcommand{\Ss}{{{\D(S)}^1}}
\newcommand{\ZZ}{{\D{Z}}}




\begin{document}

\title{	Systems of Oscillators with Piecewise Constant Coupling}
\author{Nathan Breitsch}
\date{Today}
\maketitle

\section{Introduction: Dynamical Systems and Oscillators}

\subsection
A dynamical system is a useful framework for thinking about any object in the platonic universe.  This framework equips us with the following:
-a way to represent all relevant attributes (collectively called the state) of the object
-a rule for determining all future states from an initial state

To make this idea more concrete, consider as an object a single body moving through space.  The state of the object may be captured by 
six numbers: three to represent the body's position and three to represent the body's velocity.  Given an initial velocity, 



(periodic orbit)
If we know the state of a system at any initial time $t_0$, the rule for evolution allows us to determine uniquely the state of the system for all future times.  Sometimes 
the state of a system comes back to its starting place.  When this happens, the object will continue its behavior between $t_0$ and $t_1$ repetitively forever.

(define oscillator)
An oscillator is any system which behaves repetitively.  The period $T$ of an oscillator is the amount time it takes to perform a single "lap."  At any time $t$, we 
define the phase of the oscillator $\phi (t)$ to be the progress of the oscillator through the current cycle.

find a good book and do all the fucking definitions from there.  Right now not worth your time.

(examples and properties of oscillators)
eventually get to phase reduction:

$\frac{d \phi}{dt} = \omega$

(coupling of oscillators)






\end{document}